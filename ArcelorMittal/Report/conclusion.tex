\documentclass[main.tex]{subfiles}
\begin{document}
\chapter{Conclusion}

The vibration of Metal Plate in the Hot-Strip Galvanization process is numerically modeled in this thesis. Hamilton principle is used to derive the equation of motion in weak form. A form of Euler Lagrange formulation is used to define the velocity of the plate. Inclusion of axial velocity makes Coriolis and centripetal acceleration components appear in the equation which makes it gyroscopic.  This makes it a little different from the regular plate vibration. Both Kirchhoff Thin plate and Reissner - Mindlin Plate theories are used to derive the weak form. A QUAD4 plate element is used to model the plate using Reissner - Mindlin Plate theory and PAT element is used for Kirchhoff plate. 

In results and conclusion, both the elements are compared using various studies. In most of the cases the QUAD4 element performs better. Particular when there is point load and simply supported boundary condition of a circular plate domain. PAT element shows good convergence property for distributed load.  Contrary to statement that PAT element is the one of the best plate element in literature, the PAT element shows significant Poorer convergence than QUAD4 element. Many Mesh dependency studies were conducted. After analysis a data, A optimized mesh is created that is promising for the most cases. The solutions of this FEM program is also converges to the solution given by existing Galerkin method that is developed in ArcelorMittal. 

The  Finite Element code using MATLAB Programming language is developed. Object oriented programming style is used to code the FEM program because for it's intuitive programming approach. Most of the Inputs and outputs like Loads , Boundary condition, material properties, probes etc are programmed as user defines classes. which makes it possible to create multiple instance of classes for same simulation and also enables us to have operator overloading between the objects of these classes. Much importance is given to making the program user friendly as possible. In mean time attention is paid to make it efficient. Techniques like pre-factorization and Modal Superposition techniques drastically decreased the solution time and makes it possible to solve in real time frame.Efficiency and accuracy makes it a good competitor in numerical simulation world. Instead of directly solving them, The FEM program also output the matrices in state-space form which makes it suitable for integrating into MATLAB Simulink tools. Once solved the FEM program exports the result in POS file format.

Despite being close attention paid to the boundary condition, more work needs to be done to improve the accuracy of the program. Multiple physics are involved in Hot-Dip galvanization process like fluid structure interaction between zinc and metal plate, Plate and air interaction near the air knife, Several hundred degree Celsius temperature difference between top and bottom part of the plate and electromagnetic forces in the metal strip. Most prominent one is the interaction between air and the strip. High pressure air is blown into the metal plate which induces more vibration. In literature many people addressed this issue by directly simulating Fluid structure interaction (FSI) problem and others by including added mass effect. In future developments these could be include to included to increase the efficiency. 

QUAD4 element proven its accuracy for this case, but there other much more efficient elements like MITC family of elements exists in the literature. MITC4 element would be a suitable candidate as it\textbf{ conforming and passes Patch test}. In future work time time could be spend on efficient Plate elements.  The contact between the Plates and the Rollers is not addressed here. Including contact behavior increases the usefulness of this code, The code can easily modified to simulate other location in the same process or in different process all together, which involves moving plate. Another important issue that need to be addressed in the bending plastic deformation caused by the rollers. This will cause  bending of  plate in the air knife. This is called as 'Cross-bow' effect or '\textbf{spring back effect}' effect. This will lead to uneven coating of zinc.  There is no immediate solution is in sight for this problem. More research is needed to understand this issue and to come up with a simplified way to address it in numerical simulation. 

In overall conclusion, this FEM program is able to efficiently simulate to basic vibration of the metal plate. But in the long run, more complex behaviors need to addressed. There are still more room for increase in solution speed and accuracy. Inclusion of Modal Order Reduction technique can drastically increases the Practicality of the program in the real-time active vibration control.



\end{document}