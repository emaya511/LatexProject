\documentclass[main.tex]{subfiles}



\definecolor{codegreen}{rgb}{0,0.6,0}
\definecolor{codegray}{rgb}{0.5,0.5,0.5}
\definecolor{codepurple}{rgb}{0.58,0,0.82}
\definecolor{backcolour}{rgb}{0.95,0.95,0.92}
%Code listing style named "mystyle"
\lstdefinestyle{mystyle}
{
	backgroundcolor=\color{backcolour},   commentstyle=\color{codegreen},
	keywordstyle=\color{magenta},
	numberstyle=\tiny\color{codegray},
	stringstyle=\color{codepurple},
	basicstyle=\footnotesize,
	breakatwhitespace=false,         
	breaklines=true,                 
	captionpos=b,                    
	%   keepspaces=true,                 
	%numbers=false,                    
	%numbersep=5pt,                  
	showspaces=false,                
	showstringspaces=false,
	showtabs=false,                  
	tabsize=2
}
\lstset{style=mystyle}


%\lstset{language=Matlab,%
%    %basicstyle=\color{red},
%    breaklines=true,%
%    morekeywords={matlab2tikz},
%    keywordstyle=\color{blue},%
%    morekeywords=[2]{1}, keywordstyle=[2]{\color{black}},
%    identifierstyle=\color{black},%
%    stringstyle=\color{mylilas},
%    commentstyle=\color{mygreen},%
%    showstringspaces=false,%without this there will be a symbol in the places where there is a space
%    %numbers=left,%
%    %numberstyle={\tiny \color{black}},% size of the numbers
%    %numbersep=9pt, % this defines how far the numbers are from the text
%    emph=[1]{for,end,break},emphstyle=[1]\color{red}, %some words to emphasise
%    %emph=[2]{word1,word2}, emphstyle=[2]{style},    
%}







\begin{document}
\chapter{User Guide for FEM MATLAB Code}

This Guide is intended to provide basic instruction to use to PLATE\_FEM finite element code. The Basic structure of the PLATE\_FEM is based on the Object Oriented Programming type where it contains many sub-classes inherited from few super-classes. All the class definitions are defined in a single folder named "PLATE\_FEM". To access these classes, Either run the script file in same folder or it can be placed in a MATLAB paths. In case of creating a user defined MATLAB path, use the option "Home - Set Path - Add folder" to locate the "PLATE\_FEM" folder.

\subsubsection{Prerequisite files}
Mesh file and Material Properties are a must to run the FEM program. Mesh file is generated by GMSH pre-processing tool with the file format '.msh'. For the Material property input user type file must be placed in the working directory. It is recommended to use to file name "Mat.dat.txt". The content of a example MatDat file is given below.  It is highly insisted to follow this format. Currently, Linear Isotropic material is only supported.

\VerbatimInput{codes/Mat.dat.txt}
%\lstinputlisting{codes/Mat.dat.txt}

 
\subsubsection{Initiating the program}

%\lstinputlisting{codes/Dyna.m}
Sine Plate\_FEM code generate many output files It is advised to create a separate a folder for each simulation an d place all the prerequisite  files in the same folder or any of the sub folders. Run the Matlab script file from this working folder. It is highly advised to clear all pre-exisiting data.
\begin{lstlisting}[language=matlab]
clear;
clc;
\end{lstlisting}

To initiate the study, create a object of the class that corresponds to the indented study. STATIC, MODAL and DYNAMIC are the different existing simulation types. All of them inherits from the super class PLATE\_FEM. 
\begin{lstlisting}[language=matlab]
FEM=DYNAMIC("studyname");	    % To create a  dynamic study
FEM=MODAL("studyname");	                % To find a natural frequency
FEM=STATIC("studyname");	            % To solve static problem
\end{lstlisting}
If the script is re-run with same "studyname", it will overwrite all existing data. To prevent overwriting either change the "studyname" or run it in a different folder. In order to import the material data file and mesh file use to code below.
\begin{lstlisting}[language=matlab]
FEM.ReadMesh("strip_tri.msh");      % Mesh File Input
FEM.Mat(1)=MATDat("Mat.dat.txt");   % Material file name input
\end{lstlisting}
If the ReadMSH file is exceuted without a file name it will look for a mesh with "studyname". If nothing is found it will throw an exception.
Multiple objects (FEM.Mat(1) $\cdots$ FEM.Mat(n)) of the class ( MATDat) can be created but currently only one material can be successfully applied to the element. Multiple material in a single domain can be applied but it is not fully tested. 

\subsubsection{Loads and Boundary condition definition}

$Dirichlet$ class is used to used to impose boundary displacement. It has two sub-classes namely DirichletOnPhyEn and DirichletOnNode. Boundary condition definition must be given using objects of either of the sub-classes. Basic layout of input is given below
\begin{lstlisting}[language=matlab]
FEM.Up(iu)  =DirichletOnPhyEn(FEM, PhyEn, NodalDOFS, value  );
FEM.Un(iu)  =DirichletOnNode (FEM, Nodes, NodalDOFS, value );
\end{lstlisting}
   
FEM = Object of the class PLATE\_FEM \\
Up  = Object of DirichletOnPhyEn class \\
Un  = Object of DirichletOnNode class \\
PhyEn = Physical entities values eq: [2] or [1 2 3] \\
NodalDOFS = To set the degree of freedom for which the displacement is imposed \\
value  = displacement value if value = 0 fixed and value $\neq$ 0 imposed displacement \\
*** more info **

** also add for neumann ** 

\subsubsection{Special commands for DYNAMIC class}

Sine dynamic problems require additional information to be able solve. Certain commands exist specially for DYNAMIC class. One such important command is 
\begin{lstlisting}[language=matlab]
FEM.T=FEMTime(0.01,1);	%Solution time 1s,Time step size 0.01 
\end{lstlisting}
FEMTime class is like a solver clock. In here we input the time step size and solution time. This class is created such that any modification with in the class will not affect any other functions of the program. So that in the future, automatic time stepping algorithms can be included.




\end{document}